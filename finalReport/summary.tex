\chapter{项目总结}

RUCM是一种结构化和模板化的需求规格,引入了流程、结构化句型和流程控制机制。本项目以RUCM编辑器产生的rucm文件作为输入,依据课堂所讲授的RUCM规范指定相应的规则,并按照规则来自动检查一个具体的需求违反了哪些规则,同时能够支持规则的设置。并在第二版的迭代过程中,根据老师的指导加入了中文支持与用户规则编辑、图形化前端界面等功能,形成了完整可靠的初代产品。


本项目大致上可以分为四个模块,加载.rucm文件与规则文件模块,自然语言处理模块,检查模块,规则模板设计模块。加载模块基本完成了加载文件的功能,但是尚存在许多不确定因素没有加以测试,在目前的测试样例中,能够完成正确加载文件的功能。自然语言处理模块由于其自身具有不确定性,无法保证完成了解析句子成分的功能,但是在现有测试中,能够对绝大多数输入实现正确的划分,基本实现了功能。检查模块实现了根据规则.rucm文件的功能,在当前测试中表现良好。规则设计模板模块的功能是给用户提供详尽的规则制定模板以及相应的说明书,说明书尚不完善,模板已经提供,还需完善。总体上来说,各个模块都实现了基本的功能,项目基本实现了需求分析阶段中提出的功能性需求,实现与设计阶段的一致性基本满足。在实现过程中,由于各个模块的接口不一致、类的约束不明确等一系列原因,出现了各个模块之间变量命名不一致、功能无法实现等问题。通过返回设计阶段,完善设计,解决了这些问题。