·\chapter{rule-template(规则模板)说明}
\begin{itemize}
    \item
    rule文件有一个dict组成,其中有两个字段,分别是\texttt{default}和\texttt{user-def}。分别代表默认规则列表和用户定义规则列表。
    \item
    default字段下的数据尽量不要改,唯一可以改的是status字段,当status为false是,规则处于禁用状态
    \item
    用户规则列表由用户规则聚合而成,一个用户规则下的字段包括:
    \item
    \texttt{id}: 用户规则id,不能重复
    \item
    \texttt{applyScope}:
    规则作用域,取值包括``actionStep''(flow中的句子),``allSentence''(RUCM中所有句子)
    
    \item
    \texttt{simpleRules}:
    一个用户规则可以被看做多个简单规则的聚合,简单规则字典信息包括:
        \begin{itemize}
        \item
            \texttt{subject}: 规则要检查的对象,他的取值包括:
        \item
            subject\_Val: 主语的值
        \item
            subject\_count:主语的数量
        \item
            object\_Val: 宾语的值
        \item
            object\_count: 宾语的数量
        \item
            verb\_count : 动词的数量
        \item
            verb\_tense:动词的时态
        \item
            strs: 句子中所有的词
        \item
            modal\_verb\_count:情态动词的数量
        \item
            adverb\_count:副词的数量
        \item
            pronoun\_count:代词的数量
            \texttt{operation}:
            对检查对象的约束,可以是in/notin/le/lt/eq/neq/ge/gt,分别表示约束对象在/不在/小于等于/小于/等于/不等于/大于等于/小于
        \item
            \texttt{val}: 约束列表
        \end{itemize}
            例如\texttt{\{"subject":"subject\_Val","operation":"in","val":{[}"system"{]}\}}表示主语中不能出现system字眼
            
            列表中除了出现正常字符串以外,还可以出现``\$actor''代表RUCM中的actor

        \begin{itemize}
    \item
    注意:当subject字段为XX\_count的时候,val字段中只能出现数字,当operation为除了innotin以外的字段时候,val字段中只能出现数字。
    \end{itemize}

    \item
    \texttt{status}:当status为false时,规则处于禁用状态
    \item
    \texttt{operation}:综合简单规则的逻辑操作

    \begin{itemize}
    \item
        第一个字符代表对整个检查结果的操作``-''代表无操作,``!''代表取非操作
    \item
        第一个字符往后分别代表对于简单规则之间的逻辑操作,``\&''代表与操作,``\textbar{}''代表或操作
    \end{itemize}
    \item
    description:规则描述
\end{itemize}