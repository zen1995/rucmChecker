\chapter{一致性验证}
一致性验证的任务是检查各图之间的名称、关系的一致性以及图与代码之间的一致性。

\section{OCL控制}
在本项目设计与开发的过程中,我们引入了OCL方式描述的UML模型的一致性规则,并在开发的过程中通过遵循OCL控制的方式来保证设计与实现过程中的一致性。在完成开发后,我们再一次分阶段进行了全面的一致性检验工作。

\section{第一次一致性检验}

\subsection{代码与图之间的一致性}
\begin{table}[htbp]
    \begin{tabular}{|c|c|l|}
    \hline
    不一致的类       & 不一致情况频率 & \multicolumn{1}{c|}{类图修改方案}                 \\ \hline
    Flow        & 6       & 删除了include、generation、extend 方法及属性,在实现中没有用到 \\ \hline
    Step        & 1       & 添加了parse\_step私有方法                          \\ \hline
    Sentence    & 1       & 添加了parseSentence私有方法                        \\ \hline
    NatureYype  & 2       & 添加了else\_,删除了重复属性                           \\ \hline
    RUCMLoade   & 5       & 修改了返回值,添加了两个私有方法                            \\ \hline
    RuleSubject & 1       & 删除了两个属性,在实现中没有用到                            \\ \hline
    \end{tabular}
    \end{table}
\subsection{时序图、活动图与类图之间的一致性}
除部分名称从类图中的英文转变为时序图中的中文外,并未发现不一致性。
\subsection{时序图、活动图与代码之间的一致性}
各个图与代码之间基本符合,但是时序图相较于代码过于简略,没有很好的呈现出代码的流程。


\section{第二次一致性检验}
在经过第一答辩评审后,结合中文适配、GUI界面设计、用户规则抽象等新内容的引入,在完成开发后我们做了第二次一致性检验。


在一致性检验的过程中事实上发现了不一致的地方,我们分析原因主要是在开发过程中遇到实际实现难度比较大的设计方法时采取其它不符合设计模型的方法实现所导致的,并在迭代的过程中通过为设计模型的微调保证一致性。

\subsection{代码与图之间的一致性以及各图之间的一致性}
\begin{table}[htbp]
	\centering
	\begin{tabular}{|c|c|l|}
		\hline
		检查项目       &    检查结果              \\ \hline
		类图与代码之间的一致性        & 一致   \\ \hline
		其余图与代码之间的一致性        &      GUI状态图与代码实现有出入,缺少状态转移,部分状态转移的事件不确切           \\ \hline
		各图之间是否一致   &   一致                    \\ \hline                     
	\end{tabular}
	\caption{代码开发与设计之间的一致性检验}
\end{table}
\subsection{设计与需求的一致性}
\begin{table}[htbp]
	\centering
	\begin{tabular}{|c|c|l|}
		\hline
		需求       &    检查结果              \\ \hline
		自动创建默认的RUCM规则        & 一致   \\ \hline
		规则的增删改查、保存、导入        &     一致       \\ \hline
		规则说明书   &   一致                    \\ \hline                    
		提取输入的规则JSON文件中的字段   &   一致                    \\ \hline   
		提供一系列规则管理的逻辑操作   &   一致                    \\ \hline   
		规则检查   &   一致                    \\ \hline   
		报告生成   &   一致                    \\ \hline   
	\end{tabular}
	\caption{设计与需求的一致性检验结果}
\end{table}
