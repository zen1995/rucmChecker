\chapter{一致性验证}
一致性验证的任务是检查各图之间的名称、关系的一致性以及图与代码之间的一致性检验。
\section{代码与图之间的一致性}
\begin{table}[htbp]
    \begin{tabular}{|c|c|l|}
    \hline
    不一致的类       & 不一致情况频率 & \multicolumn{1}{c|}{类图修改方案}                 \\ \hline
    Flow        & 6       & 删除了include、generation、extend 方法及属性,在实现中没有用到 \\ \hline
    Step        & 1       & 添加了parse\_step私有方法                          \\ \hline
    Sentence    & 1       & 添加了parseSentence私有方法                        \\ \hline
    NatureYype  & 2       & 添加了else\_,删除了重复属性                           \\ \hline
    RUCMLoade   & 5       & 修改了返回值,添加了两个私有方法                            \\ \hline
    RuleSubject & 1       & 删除了两个属性,在实现中没有用到                            \\ \hline
    \end{tabular}
    \end{table}
\section{时序图、活动图与类图之间的一致性}
除部分名称从类图中的英文转变为时序图中的中文外,并未发现不一致性。
\section{时序图、活动图与代码之间的一致性}
各个图与代码之间基本符合,但是时序图相较于代码过于简略,没有很好的呈现出代码的流程。